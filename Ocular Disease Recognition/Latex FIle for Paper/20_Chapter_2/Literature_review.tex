
\chapter{Literature Review}
% \graphicspath{{Chapter_2/Vector/}{Chapter_2/}}

An overview, a summary, and an assessment of the state of knowledge in a particular field of study make up a literature review. It could also highlight methodological concerns and make recommendations for further study.In this chapter, we have reviewed some papers related to our work.


\section{Ocular Disease Detection Using Advanced Neural Network Based Classification Algorithms}
Nadim and his team aimed to train a deep convolutional neural network (CNN) to detect ocular diseases using Resnet- 34,EfficientNet, MobileNetV2, and VGG-16 on the ODIR
dataset belong to 8 different classes.The accuracy of the VGG- 16 model was 97.23\%, that of the Resnet-34 model was 90.85\%, that of the MobileNetV2 model was 94.32\%, and that of the EfficientNet classification model was 93.82\%. A system for diagnosing eye diseases in real time will depend on all of these models.[16]
\section{Deep learning for ocular disease recognition: An inner-class balance}
Shakib and his team aimed to train a deep convolutional neural network (CNN) to detect ocular diseases using VGG-19 on the ODIR dataset.With VGG-19, the binary classifications were taught. The accuracy of the VGG-19 model was 98.13\% for the normal (N) vs pathological (M) myopia class, 94.03\% for the normal (N) against cataract (C), and 90.94\% for the normal (N) versus glaucoma classes (G). When the data is balanced, the accuracy of the other models likewise increases.[17]
\section{Retinal Eye Disease Detection Using Deep Learning}
Lorick Jain and his team aimed to train a deep convolutional neural network (CNN) to detect ocular diseases. The objective of this study is to automatically distinguish between photos with retinal issues and photographs of healthy retinas without conducting any specific segmentation or feature extraction. Instead, any retinal fundus picture is automatically classified as healthy or sick using a deep learning algorithm. The network’s architecture is straightforward and quick. Two datasets, including actual patient retinal fundus pictures collected from a nearby hospital, were used to evaluate the model. This model’s accuracy was determined to be between 96.5\% and 99.7\%.[18]
\section{Retinal Image Analysis for Diabetes-Based Eye Disease Detection Using Deep Learning}
This study presents an automated method for disease localization and segmentation based on the fuzzy k-means (FKM) clustering algorithm and Fast Region-based Convolutional Neural Network (FRCNN) algorithm. Since datasets don't often contain bounding-box annotations, They created them using ground truths. The FRCNN is an object identification method that needs these annotations to function. After segmenting out the annotated pictures using FKM clustering, the annotated images are then used to train the FRCNN for localisation. Through intersection-over-union processes, the segmented areas are then compared to the ground facts. They  employed the Diaretdb1, MESSIDOR, ORIGA, DR-HAGIS, and HRF datasets for performance evaluation. The effectiveness of the methodology in terms of both illness identification and segmentation is confirmed by a careful comparison against the most recent techniques. In terms of localisation, the suggested model obtained mAPs for DR, DME, and glaucoma of 0.945, 0.943, and 0.941, respectively. Backpropagation estimation is used by FRCNN, along with the addition of the bbox regression and classification heads, and multi-task loss was used to train the model. Using FRCNN, the suggested technique concurrently identifies anomalies in the DR, DME, and glaucoma areas. Finally, FKM clustering correctly removes the regions from localized regions. Our segmentation accuracy for the DR, DME, and glaucoma areas was 0.952, 0.958, and 0.9526, respectively.[19]
\section{Automated detection of mild and multi-class diabetic eye diseases using deep learning}
This study set out to develop an automated classification system that took into account two different diabetic eye disease (DED) scenarios: I mild multi-class DED and (ii) multi-class DED. Our model was evaluated using numerous datasets that an optometrist had annotated. The top two pretrained convolutional neural network (CNN) models from ImageNet were used in the experiment. Various performance-improving approaches, such as fine-tuning, optimization, and contrast enhancement, were also used. The VGG16 model achieved a maximum accuracy of 88.3\% for multi-class classification and 85.95\% for moderate multi-class classification.[20]
\section{A Deep Learning Method for Microaneurysm Detection in Fundus Images}
In order to detect MA in fundus pictures, a Stacked Sparse Autoencoder (SSAE), an example of a DL technique, is proposed in this study. The first fundus photos are used to create tiny image patches. To find distinctive properties of MA, the SSAE learns high-level features just from pixel intensities. Each image patch is classified as MA or non-MA using the high-level characteristics learned by SSAE. The training/testing data and ground truth are provided via the open benchmark DIARETDB. The 89 photos are divided into 2182 image patches with MA lesions, which serve as positive data, and 6230 image patches without MA lesions, which are produced using a randomly selected sliding window operation, which serve as negative data. SSAE learnt directly from the raw image patches and automatically retrieved the distinctive characteristics to categorize the patches using Softmax Classifier without the need for blood vessel removal or other preprocessing procedures. Using 10-fold cross-validation, the fine-tuning process resulted in an improved F-measure of 91.3\% and an average area under the ROC curve (AUC) of 96.2\%.[21]
\section{Data Driven Approach for Eye Disease Classification with Machine Learning}
The objective of this project is to create a basic framework for storing diagnostic data in an international standard format to make it easier for machine learning algorithms to anticipate illness diagnosis based on symptoms. A user-friendly interface was created in an effort to assure error-free data entering. A number of machine learning techniques, including as Decision Tree, Random Forest, Naive Bayes, and Neural Network algorithms, were also employed to assess patient data based on a variety of variables, such as age, medical history, and clinical observations. The diagnosis was established using the ICD-10 coding published by the American Academy of Ophthalmology, and the data was organized in accordance with hierarchical hierarchies created by medical specialists. Furthermore, new categories for symptoms as well as diagnoses will be added as part of the system's self-learning evolution.The classification results using tree-based approaches showed that, given enough data, the suggested framework works adequately. In contrast to more sophisticated techniques like neural networks and the naive Bayes algorithm, the random forest and decision tree algorithms' prediction rate is greater than 90\% because of a structured data arrangement.[22]