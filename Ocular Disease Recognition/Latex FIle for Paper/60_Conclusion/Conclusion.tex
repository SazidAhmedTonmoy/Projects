\chapter{CONCLUSION}

In this study, we created four neural network-based models for classifying eye diseases. These models are VGG-19, Resnet-50, and MobileNetV2. When it comes to categorizing ocular disorders from fundus pictures, the Resnet50 offered the best accuracy for all models. The other models' performance was similarly acceptable. MobilenetV2 has no issue with hundred percent training accuracy. To verify the efficacy of our suggested strategy, we conducted extensive tests using the ODIR- 2019 dataset, which is open to the general public. In comparison to the current CNN-based ocular illness classification models, our suggested technique can produce results that are more spectacular while using less computing power. The nicest thing about our suggested approach is how easy it may be applied to various kinds of illness categorization based on medical images. Such a method will change the area of visual illness diagnostics and be of enormous use to medical experts.Additionally, this work could benefit from the use of ocular image segmentation. To address the imbalance issue, research can create comparable pictures of eye illness using generative adversarial networks (GANs). Additionally, a system like this would revolutionize the field of diagnosing eye diseases and be very helpful to medical professionals. Our opinion is that it can still be a valuable model, and there will likely be possibilities to improve it with further research and study in the near future.

\section{Data Availability}
The data used to support the findings of this study are freely available at https://www.kaggle.com/andrewmvd/ocular-disease-recognition-odir5k.
