
\chapter{Introduction}


Various ocular diseases are capable of causing permanent and irreversible damage to the patient’s vision, and in extreme cases, they can even lead to blindness [1-3]. Although effective treatments are available for these ocular diseases, these treatment options can only be implemented if the disease is diagnosed as early as possible. Ocular diseases are primarily diagnosed using color fundus photography or CFP [4]. This technique is utilized in order to record the interior surface of the human eye so that various types of possible ocular diseases can be detected [5]. Although this method of diagnosis is effective, it’s still quite difficult to detect certain ocular diseases using CFP. Some of the most prevalent ocular diseases, such as cataracts, myopia, and diabetic retinopathy, are difficult to diagnose as they show very few initial symptoms. [6] Moreover, the process of manually inspecting and detecting ocular diseases is a laborious task, and this process is not that accurate [7].\\ In recent times, deep learning-based neural network models have shown promising results in medical image classification and object detection. Moreover, that is why convolutional neural network-based models have been extensively studied for ocular disease detection. Given this, it’s critical to offer these people affordable or free complete eye care treatments. Deep-learning-based algorithms are becoming more common in medical image analysis. Deep-learning-based models have been demonstrated to perform well in numerous tasks such object recognition, sentiment analysis [8], medical picture classification [9], and illness diagnosis [10]. One of the most important steps in minimizing an ophthalmologist’s workload is the automated diagnosis of diseases. Without the need for human interaction, illnesses may be detected using deep learning and computer vision technology. Only a small number of these studies have been able to fully diagnose [11] more than one eye illness, despite the fact that many of them have produced encouraging results. To accurately detect diverse eye conditions, more study is required to examine the many elements of fundus imaging [12]. This study suggests a system that uses deep learning to recognize different eye diseases. Multilabel categorization has been used as a different strategy [13]. The ocular disease’s datasets [14–15] are quite unbalanced. This imbalance makes it difficult to accurately identify or classify sickness or even a normal picture. This method is not recommended for broad classification problems due to its low accuracy.\\
As a result, our study initially balanced the dataset by training the classes on the pretrained Resnet50, VGG-19, and MobilenetV2 architecture with the same amount of data for each class. By selecting the equal number of photos for each classes, we first loaded the dataset and the associated image into the dataset. For the VGG-19, Resnet50, and Mobilenetv2 models in this work, the transfer learning approach was used. The accuracy of each class rose once we correctly balanced the dataset. The remaining portions of the essay are structured as follows: The relevant work for this study is displayed in Section 2. Dataset’s description is displayed in Section3. All the tools and techniques are extensively covered in Section 4. After discussing our study’s results and performance analysis in Section 5. Section 6 brings our work's deployment and Section 7 brings our work to a close.


