\chapter*{\centering Abstract}

\pagestyle{fancy}
\fancyhf{}
\fancyhead[LO,RE]{Abstract}
\fancyhead[LE,RO]{\thepage}

\addcontentsline{toc}{chapter}{Abstract}

Early detection and identification of eye disorders using fundus pictures is among ophthalmologists’ most difficult responsibilities. However, eye illness diagnosis by hand is challenging, time-consuming, and error-prone. For the purpose of employing fundus pictures for early identification of different ocular disorders, a computer-aided automated ocular disease detection system is necessary. Such a system can now be accomplished because to deep learning algorithms’ improved picture categorization skills. Four deep learning-based models for pinpointing ocular diseases are presented in this work. For this work, we used the ODIR dataset, which consists of 5000 fundus images. we took 3404 fundus images divided into 5 distinct groups, to train cutting-edge image classification algorithms including Resnet-50, MobileNetV2, and VGG-19.\\
\textbf{Index Terms: }Ocular Disease Classification, Color Fundus
Photography, Ocular Disease Detection, Convolutional Neural
Networks, VGG-19, Resnet-50, MobileNetV2, Deep Transfer
Learning
