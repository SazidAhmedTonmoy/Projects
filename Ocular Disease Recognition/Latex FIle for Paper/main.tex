\documentclass[12pt,a4paper,twoside,openright]{report}

%%%%%%% Import packages %%%%%%%
\usepackage{graphicx}
\usepackage{epsfig}
\usepackage{epstopdf,amsfonts,array,multirow,amsmath,tabularx,setspace,amssymb,enumerate,csquotes}
	
% \usepackage{url}
% \usepackage{bbm}
% \usepackage{eufrak}
% \usepackage{marvosym}
% \usepackage{moderncvcompatibility}

%% packages with settings
% Inline listing package
% \usepackage[inline]{enumitem}
\usepackage{listings}
% Page geometry
\usepackage{geometry}
\setlength{\evensidemargin}{0.0cm}
\setlength{\oddsidemargin}{0.0cm}
\setlength{\topmargin}{-1.27cm}
\setlength{\headheight}{32pt}
\setlength{\headsep}{1em} %% Setting for gap in header and main body%%
\setlength{\textheight}{24.0cm}
%\setlength{\textwidth}{14.65cm}
\setlength{\footskip}{1.5cm}
\setlength{\parindent}{1.5em}
\setlength{\parskip}{1em}

% Set line spacing to 1.25 because
% with 12pt font, 1.5 spacing is too large
\setstretch{1.25}

\usepackage{fancyhdr}
\usepackage[usenames,dvipsnames]{xcolor}
\usepackage{relsize}
\usepackage{bigints}

% Packages for table of contents
\setcounter{tocdepth}{3}

\usepackage[nottoc]{tocbibind}
\usepackage[titles,subfigure]{tocloft}

\makeatletter
\newlength{\@chapterlength} % Original Magic Value - RMR
\setlength{\@chapterlength}{1.5em} % This code adds space for \@chapapp for toc entries
\newcommand\@chapterheadsmark{1}
\ifnum \@chapterheadsmark = 1
\settowidth{\@chapterlength}{\@chapapp}
\addtolength{\@chapterlength}{3.8em} % Space between toc labels and values
\fi
% \def\l@chapter{\pagebreak[3]\vskip 1em plus 1pt \@dottedtocline{0}{0em}{1.5em}}
% \def\l@chapter{\pagebreak[3]\vskip 1em plus 1pt \@dottedtocline{0}{0em}{\@chapterlength}}

\def\l@section{\@dottedtocline{1}{\@chapterlength}{2.3em}}

\newlength{\@subsectionlength}
\setlength{\@subsectionlength}{\@chapterlength}
\addtolength{\@subsectionlength}{2.3em}
\def\l@subsection{\@dottedtocline{2}{\@subsectionlength}{3.2em}}

\newlength{\@subsubsectionlength}
\setlength{\@subsubsectionlength}{\@chapterlength}
\addtolength{\@subsubsectionlength}{5.5em}
\def\l@subsubsection{\@dottedtocline{3}{\@subsubsectionlength}{4.1em}}

\newlength{\@paragraphlength}
\setlength{\@paragraphlength}{\@chapterlength}
\addtolength{\@paragraphlength}{8.5em}
\def\l@paragraph{\@dottedtocline{4}{\@paragraphlength}{5em}}

\newlength{\@subparagraphlength}
\setlength{\@subparagraphlength}{\@chapterlength}
\addtolength{\@subparagraphlength}{10.5em}
\def\l@subparagraph{\@dottedtocline{5}{\@subparagraph}{6em}}
\makeatother

\renewcommand\cftchappresnum{\chaptername~}
\newlength\mylength
\settowidth\mylength{\cftchappresnum\cftchapaftersnum\qquad}
\addtolength\cftchapnumwidth{\mylength}

\usepackage[acronym,nomain,toc]{glossaries}
\makeglossaries
\usepackage[intoc]{nomencl}
\makenomenclature
\renewcommand{\nomname}{List of Symbols}
\usepackage[toc,page]{appendix}

% Packages for floats like figures and tables
\usepackage{float}
\usepackage{caption}
% \usepackage[center]{caption}
% \usepackage{longtable}
% \usepackage{subfig}
\usepackage{subfigure}
% \usepackage{subcaption}
% \usepackage[chapter]{algorithm}
\usepackage{algpseudocode}
% \usepackage{algorithmic}

\usepackage[base]{babel}
\usepackage{lipsum,blindtext} % just to generate text for the example

\usepackage{color}   %May be necessary if you want to color links

\usepackage{datetime}
\newdateformat{thisdate}{%
	\twodigit{\THEDAY}}
\newdateformat{thismonthdigit}{%
	\twodigit{\THEMONTH}}
\newdateformat{thismonth}{%
	\monthname[\THEMONTH]}
\newdateformat{thisyear}{%
	\THEYEAR}

% \usepackage{cite}
% \usepackage{notoccite}
\usepackage[backend=biber, style=ieee, sorting=none, defernumbers=true]{biblatex}
\addbibresource{bibliography.bib}

% Hyperref should be the last package to be loaded
\usepackage{hyperref}
% \hypersetup{
% 	breaklinks=true,
%     colorlinks=true, %set true if you want colored links
%     linktoc=all,     %set to all if you want both sections and subsections linked
%     linkcolor=black,  %choose some color if you want links to stand out
%     citecolor=black,
%     urlcolor=black
% }

\hypersetup{linktoc=all, colorlinks=true, linkcolor=Black, citecolor=teal, urlcolor=red}


\DeclareMathOperator{\arccot}{arccot}
\DeclareMathOperator{\sinc}{sinc}
\DeclareMathOperator*{\argmin}{arg\;min}
\DeclareMathOperator*{\argmax}{arg\;max}

\newtheorem{theorem}{Theorem}[chapter]
\newtheorem{proposition}{Proposition}[chapter]
\newtheorem{lemma}{Lemma}[chapter]
\newtheorem{definition}{Definition}[chapter]
\newtheorem{prop}{Proposition}
%-------------------------------------------------------------------

%\flushbottom
%--------------------------------------------------------------------------
\raggedbottom                               % (Remove Unnecessary space b/w paragraphs)
%--------------------------------------------------------------------------
\begin{document}

% --------- Title and abstract etc in the front matter ----------- %
\pagenumbering{gobble}
% TITLE PAGE
\begin{titlepage}
\begin{center}

{\fontencoding{T1}\fontfamily{pzc}\fontseries{m}\fontshape{n}\selectfont
\huge \bfseries Tooth Decay Detection Using Different YOLO Algorithms}\\
\vspace*{1cm}

{\large  \itshape CSE498R report submitted in partial fulfillment of the requirements for the degree}\\
\vspace*{0.5cm}
{\large \itshape {of}}\\
\vspace*{0.5cm}

{\fontencoding{T1}\fontfamily{pzc}\fontseries{m}\fontshape{n}\selectfont
\huge \bfseries Bachelor of Science in Computer Science and Engineering}\\
\vspace*{0.5cm}

{\large \itshape {by}}\\
\vspace*{0.5cm}
\begin{center}

{\large \bfseries Md Sazid Ahmed Tonmoy }\\
{\large \bfseries ID: 1911498042}\\
\par\end{center}
\vspace*{0.5cm}

{\large Under the guidance of\\}
\vspace*{0.2cm}

{\Large \bfseries Dr. Sifat Momen\\}
{\Large \bfseries Associate Professor\\}

\vspace*{1.0cm}

\includegraphics[scale=0.6]{nsu.png}
\vspace*{0.5cm}

\textsc{ \bfseries{DEPARTMENT OF ELECTRICAL & COMPUTER ENGINEERING\\
NORTH SOUTH UNIVERSITY\\Bashundhara, Dhaka-1229, Bangladesh\\}}
{ \bfseries Summer {\thisyear\today}\\}
\vspace*{0.1cm}

\end{center}
\end{titlepage}




%%%%%%%%%%%% Front Matter %%%%%%%%%%%%
\pagenumbering{roman}
% Page 5 : Approval of the DAC

% \chapter*{} is needed to link the hyperlinks in contents to the corresponding pages properly.
\chapter*{}
\vspace*{-5cm}
% \thispagestyle{empty}

{\centering \includegraphics[width=\linewidth, scale=1]{01_Declaration/letterhead.png}}
{\hrule width \hsize height 1.5pt \kern .5mm \hrule width \hsize height 1.5pt}
\vspace*{2ex}
\begin{center}
\textbf{\Large DECLARATION}
\end{center}
\vspace{2ex}
It is hereby acknowledged that: 
\begin{enumerate}[a.]
	\item No illegitimate procedure has been practiced during the preparation of this document.
	\item This document does not contain any previously published material without proper citation.
	\item This document represents our own accomplishment while being Undergraduate Students in the North South University.
\end{enumerate}
Sincerely,\\
\vspace{10em}
\begin{minipage}[t]{0.35\textwidth}
\end{minipage}%
\hfill
\begin{minipage}[t]{0.55\textwidth}
    \raggedleft
    \hrule\vspace{2ex}
     \textbf{Md Sazid Ahmed Tonmoy} \\
     \textbf{ID: 1911498042}
\end{minipage}
\addcontentsline{toc}{chapter}{Declaration}
\let\cleardoublepage\clearpage


% Page 7: Certificate on department letterhead

\chapter*{}
\vspace*{-5cm}
% \thispagestyle{empty}
% {~} % Buffer blank character to prevent horizontal underflow due to next \vspace
% \vspace{10em} % Increase if the dept. letterhead requires more space


{\centering \includegraphics[width=\linewidth, scale=1]{02_Approval/letterhead.png}}
{\hrule width \hsize height 1.5pt \kern .5mm \hrule width \hsize height 1.5pt}

\vspace*{2ex}
\begin{center}
\textbf{\Large Approval}
\end{center}

\par This is to certify that the cse499 report entitled Ocular Disease Recognition through Deep Learning Architectures, submitted by Md Sazid Ahmed Tonmoy (ID: 1911498042) and Hawlader Md Hadiuzzaman Reaz (ID: 1821940042), are undergraduate students of the Department of Electrical & Computer Engineering, North South University. This report partially fulfil the requirements for the degree of Bachelor of Science in Computer Science and Engineering on September 11, 2022, and has been accepted as satisfactory.

\vspace{4em}
\begin{minipage}[t]{0.35\textwidth}
    Date: \\
\end{minipage}%
\hfill
\begin{minipage}[t]{0.55\textwidth}
    \raggedleft
    \hrule\vspace{2ex}
     \textbf{MR. ABU OBAIDAH} \\
    Lecturer \\
 	Department of Electrical & Computer Engineering \\
 	North South University \\
 	Dhaka, Bangladesh
	

     \vspace{5em}
     \hrule\vspace{2ex}
     \textbf{DR. RAJESH PALIT} \\
    Professor and Chair \\
 	Department of Electrical & Computer Engineering \\
 	North South University \\
 	Dhaka, Bangladesh
\end{minipage}

\addcontentsline{toc}{chapter}{Certificate}

\let\cleardoublepage\clearpage

 
%%%%%%%%%%%% Abstract %%%%%%%%%%%%
\setstretch{2}  % double spacing
\chapter*{\centering Abstract}

\pagestyle{fancy}
\fancyhf{}
\fancyhead[LO,RE]{Abstract}
\fancyhead[LE,RO]{\thepage}

\addcontentsline{toc}{chapter}{Abstract}

Early detection and identification of eye disorders using fundus pictures is among ophthalmologists’ most difficult responsibilities. However, eye illness diagnosis by hand is challenging, time-consuming, and error-prone. For the purpose of employing fundus pictures for early identification of different ocular disorders, a computer-aided automated ocular disease detection system is necessary. Such a system can now be accomplished because to deep learning algorithms’ improved picture categorization skills. Four deep learning-based models for pinpointing ocular diseases are presented in this work. For this work, we used the ODIR dataset, which consists of 5000 fundus images. we took 3404 fundus images divided into 5 distinct groups, to train cutting-edge image classification algorithms including Resnet-50, MobileNetV2, and VGG-19.\\
\textbf{Index Terms: }Ocular Disease Classification, Color Fundus
Photography, Ocular Disease Detection, Convolutional Neural
Networks, VGG-19, Resnet-50, MobileNetV2, Deep Transfer
Learning



%%%%%% Table of Contents %%%%%%
%%%%%%% To create 1.5 spacing in contents
\setstretch{1.5}

%%%%%%%%%% To rename Table of Contents
\renewcommand{\contentsname}{Table of Contents}
% \renewcommand{\contentsname}{Contents}
%%%%%%%%%%
\tableofcontents
\addcontentsline{toc}{chapter}{\contentsname}
\pagestyle{fancy}
\fancyhf{}
\fancyhead[LO,RE]{\contentsname}
\fancyhead[LE,RO]{\thepage}

\listoftables
\listoffigures
\pagestyle{fancy}
\fancyhf{}
\fancyhead[LO,RE]{List of Tables}
\fancyhead[LE,RO]{\thepage}



% %%%%%% List of Algorithms %%%%%%
% \chapter*{}
% \addcontentsline{toc}{chapter}{List of Algorithms}
% \listofalgorithms
% \newpage



\let\cleardoublepage\clearpage


\pagenumbering{arabic}
%%%%%%%% To create double spacing in chapters
\setstretch{2}

% Fancy header
\pagestyle{fancy}
\renewcommand{\sectionmark}[1]{\markright{\thesection~#1}{}}
\renewcommand{\chaptermark}[1]{\markboth{\chaptername~\thechapter~-~#1}{}}
\fancyhf{}
%\fancyhead[RE]{\leftmark}
%\fancyhead[LO]{\rightmark}
\fancyhead[LE,RO]{\thepage}

%%%%%% Chapters %%%%%%
%% Remove any command starting with "\lipsum"
%% from the chapter text. They are used to generate
%% junk test paragraphs


\chapter{INTRODUCTION}


Tooth decay is one of the most widespread health problems in the world. It occurs when a tooth's enamel is damaged. Cavities in the teeth caused by tooth decay can lead to tooth loss. According to a recent study, tooth decay has surpassed heart disease as the most frequent health problem worldwide, affecting almost 34.1\% of the population [2]. People in many parts of the world have limited access to dental specialists. Since caries is not life threatening, many patients with untreated caries wait until it is too late, when major complications have already occurred and treatment is too expensive. Dental cavities that go untreated can lead to pulpitis and periapical disorders [3]. However, tooth decay can be halted and the decay process reversed if diagnosed early enough. The enamel has a self-healing property. As a result, early detection of tooth decay is an important factor of treatments to prevent dental caries. It may make dental treatment more affordable for persons with lower and medium incomes.\\
Artificial Intelligence (AI) is becoming increasingly popular and widely used in medicine to diagnose and treat patients more rapidly and accurately. Deep learning (DL), an artificial intelligence (AI) method, has been used to automate decision-making processes in numerous clinical dental situations in recent years [4]. By autonomously learning from datasets containing human annotations from dental specialists, the approach, which consists of multi layer ConvNets, has already showed promising accuracy on unforeseen data.\\
The focus of this study was on identifying tooth decay in three phases. The three phases are visible change without cavitation, visible change with micro-cavitation, and visible change with cavitation. Each phase is distinct in terms of personality, patterns, and shapes. The characteristics of the stages are described as follows and illustrated in Figure 1:
\begin{enumerate}
\item Visible change without cavitation: This is the earliest stage of tooth decay, when a lesion forms on the tooth. It causes a slight darkening of the tooth's surface, which is usually white or brown [1].
\item Visible change with micro-cavitation: In this stage, demineralization continues and the tooth enamel (the uppermost layer of the tooth's structure) begins to break down.
\item Visible change with cavitation: In this stage, the dentin layer of the tooth is impacted as the tooth decay proceeds. Bacteria get inside the decaying pulp and causes infection. 
\end{enumerate}
\vspace{5pt}
\begin{figure}[H]
    \centering
    \includegraphics[scale=0.5]{10_Chapter_1/1.jpg}
    \caption{(a) Visible change without cavitation, (b) Visible change with micro-cavitation, (c) Visible change with cavitation}
    \label{Types of class}
\end{figure}
This paper offers a method for accurately predicting and classifying the three phases of tooth decay using deep learning techniques. For the automatic detection of dental decay from oral photos, we constructed a deep Convolutional Neural Network. The model classifies the presence of dental decay in a given image and uses bounding boxes to locate the findings. 
We used three different YOLO object detection models to train the dataset. In section V, the comparison study of the three has been analyzed for a better representation and understanding of the trained model's efficiency and accuracy. The rest of the paper is set out as follows: The second section covers relevant works. Section III dives into the model's training phases. Experimental setup is covered in Section IV. Section VI concludes with some ideas for the future.


\newpage
%\newpage\null\newpage

\chapter{LITERATURE REVIEWS}
% \graphicspath{{Chapter_2/Vector/}{Chapter_2/}}

An overview, a summary, and an assessment of the state of knowledge in a particular field of study make up a literature review. It could also highlight methodological concerns and make recommendations for further study.In this chapter, we have reviewed some papers related to our work.


\section{Deep Learning-Based Dental Plaque Detection on Primary Teeth: A Comparison With Clinical Assessments} Wenzhe You and his team aimed to train a deep convolutional neural network (CNN) to detect caries lesions using DeepLabV3+.  886 intraoral images of primary teeth were utilized to train a conventional neural network (CNN) architecture. 98 intraoral pictures of primary teeth were evaluated by the AI model to verify clinical viability. Additionally, digital camera images of teeth were taken. A skilled pediatric dentist evaluated the images and noted the plaque-containing areas. The plaque-containing sites were then detected after the application of a plaque-disclosing agent. To assess the consistency of the manual diagnosis, the dentist drew the plaque area on the 98 digital camera photographs once again after one week. To assess the diagnostic effectiveness of each method based on lower-resolution pictures, 102 intraoral photographs of primary teeth were annotated to signify the plaque regions acquired by the AI model and the dentist. The degree of detection accuracy was measured using the mean intersection-over-union (MIoU) metric.\\
To start, they used the visual object classes dataset to pretrain the fundamental DeepLab network and derive the initial weights using transfer learning methods. Second, they used their picture dataset of primary teeth, which includes images of 886 primary teeth before and after employing a dental plaque-disclosing agent, to train a DeepLabV3+ model. To enable the AI model to compare the findings and learn from its errors, the dental plaque detected by the AI model was compared with the actual dental plaque regions.\\
The MIoU for recognizing plaque on the examined dental pictures was 0.726 ± 0.165. The dentist's MIoU when first diagnosing the 98 photographs obtained by the digital camera was 0.695 ± 0.269 and 0.689 ± 0.253 after one week. The AI model had a higher MIoU (0.736 ± 0.174) than the dentist, and the outcomes were unchanged after one week. The MIoU was 0.652 ± 0.195 for the dentist and 0.724 ± 0.159 for the AI model when they evaluated the 102 intraoral pictures. A paired t-test revealed no statistically significant difference between the human expert and AI model in the ability to identify dental plaque on primary teeth (P >.05). [5]\\

\section{Deep Learning Application in Dental Caries Detection Using Intraoral Photos Taken by Smartphones} Mai Thi Giang Thanh and his team worked with Intraoral Photos Taken by Smartphones to train several deep convolutional neural network (CNN) to detect dental caries. The objective of this work was to use a deep learning system to diagnose the phases of smooth surface caries using photos from a smartphone. Materials and procedures A training dataset of 1902 images of teeth's smooth surface acquired using an iPhone 7 by 695 individuals was used. To identify early caries lesions and cavities, four deep learning models—You Only Look Once version 3 (YOLOv3), RetinaNet, Faster Region-Based Convolutional Neural Networks (Faster R-CNNs), and Single-Shot Multi-Box Detector (SSD)—were examined. The International Caries Categorization and Management System (ICCMS) classification of a dentist's diagnosis based on an image inspection served as the reference standard.\\
The two evaluated models with the highest sensitivity for cavitated caries were YOLOv3 and Faster R-CNN, with 87.4\% and 71.44\%, respectively. For visibly non-cavitated samples, these two models' sensitivity levels were only 36.9\% and 26\%, respectively (VNC). For cavitated caries and over 71\% for VNC, the specificity of the four models was above 86\%. For the clinical diagnosis of dental caries using smartphone photos, the YOLOv3 and Faster R-CNN models showed promise. The new work offers a rudimentary understanding of how AI may be applied in clinical settings once it has been developed in the lab. [6]\\

\section{Detection and Diagnosis of Dental Caries Using a Deep Learning-Based Convolutional Neural Network Algorithm} Jae-Hong Lee and his team worked with dental images to evaluate the efficacy of deep CNN algorithms for detection and diagnosis of dental caries on periapical radiographs. A training and validation dataset (n = 2400 [80\%]) and a test dataset (n = 600 [20\%]) were created from a total of 3000 periapical radiography images. For preprocessing and transfer learning, a GoogLeNet Inception v3 CNN network that has already been trained was employed. For detection and diagnostic performance of the deep CNN algorithm, the diagnostic accuracy, sensitivity, specificity, positive predictive value, negative predictive value, receiver operating characteristic (ROC) curve, and area under the curve (AUC) were calculated. \\
Premolar, molar, and premolar and molar models all had diagnostic accuracies of 89.0 percent (80.4-93.3), 88.0 percent (79.2-93.1), and 82.0 percent (75.5-87.1), respectively. The AUC on premolar, molar, and combined premolar and molar models for the deep CNN method was 0.917 (95 percent CI 0.860-0.975), 0.890 (95 percent CI 0.819-0.961), and 0.845 (95 percent CI 0.790-0.901). The best AUC was produced by the premolar model, which was far better than that for the other models (P 0.001). This study showed how deep CNN architecture might be useful for identifying and diagnosing dental cavities. In periapical radiographs, a deep CNN algorithm significantly improved performance in identifying dental caries.[7]\\
\section{Automated Dental Cavity Detection System Using Deep Learning and Explainable AI}
An artificial intelligence system that identifies the existence of dental cavities on photos and visually explains each diagnostic was created to manage tooth cavities. The technique used in this work identifies cavities on pictures of several teeth and four tooth surfaces, unlike earlier systems that could only detect cavities on one removed tooth with one exposed tooth surface. 506 de-identified photos from web sources and willing human subjects were gathered for training. A ResNet-27 design was found to be the most effective using curriculum learning, reaching 82.8\% accuracy and 1.0 in sensitivity. The system's diagnosis might also be visually explained using Local Interpretable Model Agnostic Explanation. This technology can clearly explain its diagnosis to users, which is an essential ability used by dentists.[8]
\section{PaXNet: Dental Caries Detection in Panoramic X-ray using Ensemble Transfer Learning and Capsule Classifier}
The proposed model benefits from various pretrained deep learning models through transfer learning to extract relevant features from x-rays and uses a capsule network to draw prediction results. On a dataset of 470 Panoramic images used for features extraction, including 240 labeled images for classification, their model achieved an accuracy score of 86.05\% on the test set. As long as the difficulties of employing Panoramic x-rays of actual patients are taken into consideration, the resultant score reveals satisfactory detection performance and an increase in caries detection time. Their model achieved recall scores of 69.44\% and 90.52\% for moderate and severe caries lesions, respectively, in the test set of photos containing caries lesions, demonstrating that it is easier to detect severe caries spots and that effective mild caries identification requires a more robust and bigger dataset. This work is a step toward creating a completely automated effective decision support system to help domain specialists, especially in light of the originality of the current research study's use of panoramic photographs.[9]\\
\begin{table}[H}
\caption{Literature Review}
\centering
\resizebox{\textwidth}{!}{\begin{tabular}{|l|l|l|l|l|l|} 
\hline
Work                                                                                                                                                                      & Dataset Features                                                                                                                                                                              & Dataset Size                                                                                                   & \begin{tabular}[c]{@{}l@{}}Algorithm \\Selection\end{tabular}                                              & Evaluation                                                                       & Results                                                        \\ 
\hline
\multirow{3}{*}{\begin{tabular}[c]{@{}l@{}}Deep learning Based Dental Plaque\\Detection on Primary Teeth: A Comparison\\with Clinical Assessments\end{tabular}}           & Digital camera images of teeth,                                                                                                                                                               & \multirow{3}{*}{\begin{tabular}[c]{@{}l@{}}886 training data , \\102 testing data\end{tabular}}                & \multirow{3}{*}{DeepLabV3+}                                                                                & \multirow{3}{*}{\begin{tabular}[c]{@{}l@{}}\\(MIoU) metric.\end{tabular}}        & \multirow{3}{*}{0.724 ±~0.159}                                 \\
                                                                                                                                                                          & \begin{tabular}[c]{@{}l@{}}Annotated plaque area, \\Dataset evaluated both \\human expert and AI\end{tabular}                                                                                 &                                                                                                                &                                                                                                            &                                                                                  &                                                                \\
                                                                                                                                                                          &                                                                                                                                                                                               &                                                                                                                &                                                                                                            &                                                                                  &                                                                \\ 
\hline
\multirow{4}{*}{\begin{tabular}[c]{@{}l@{}}Deep Learning Application in Dental Caries\\Detection Using Intraoral Photos \\Taken by Smartphones\end{tabular}}              & \begin{tabular}[c]{@{}l@{}}Smartphone photos \\taken with Iphone7,\end{tabular}                                                                                                               & \multirow{4}{*}{1902 training dataset}                                                                         & \multirow{4}{*}{\begin{tabular}[c]{@{}l@{}}YOLOv3, \\RetinaNet, \\Faster \\R-CNNs, \\and SSD\end{tabular}} & \multirow{4}{*}{\begin{tabular}[c]{@{}l@{}}Accuracy, \\Sensitivity\end{tabular}} & \begin{tabular}[c]{@{}l@{}}Highest \\Accuracy\end{tabular}     \\
                                                                                                                                                                          & \begin{tabular}[c]{@{}l@{}}Images of teeth’s \\smooth surface, \\Annotated area\end{tabular}                                                                                                  &                                                                                                                &                                                                                                            &                                                                                  & \begin{tabular}[c]{@{}l@{}}YOLOv3 \\74\%\end{tabular}          \\ 
\cline{6-6}
                                                                                                                                                                          &                                                                                                                                                                                               &                                                                                                                &                                                                                                            &                                                                                  & \begin{tabular}[c]{@{}l@{}}Highest \\Sensitivity\end{tabular}  \\
                                                                                                                                                                          &                                                                                                                                                                                               &                                                                                                                &                                                                                                            &                                                                                  & \begin{tabular}[c]{@{}l@{}}R-CNN \\87.4\%\end{tabular}         \\ 
\hline
\multirow{6}{*}{\begin{tabular}[c]{@{}l@{}}Detection and Diagnosis of Dental\\Caries Using a Deep Learning \\Based Convolutional Neural Network \\Algorithm\end{tabular}} & \multirow{6}{*}{\begin{tabular}[c]{@{}l@{}}Images of dental caries \\on periapical radiographs, \\Annotated area, \\Premolar, molar, and \\premolar and \\molar are the classes\end{tabular}} & \multirow{6}{*}{\begin{tabular}[c]{@{}l@{}}3000 images, \\2400 training data, \\600 testing data\end{tabular}} & \multirow{6}{*}{\begin{tabular}[c]{@{}l@{}}GoogLeNet \\Inception v3\end{tabular}}                          & \multirow{6}{*}{Accuracy}                                                        & Premolar                                                       \\
                                                                                                                                                                          &                                                                                                                                                                                               &                                                                                                                &                                                                                                            &                                                                                  & 89\%                                                           \\ 
\cline{6-6}
                                                                                                                                                                          &                                                                                                                                                                                               &                                                                                                                &                                                                                                            &                                                                                  & Molar                                                          \\
                                                                                                                                                                          &                                                                                                                                                                                               &                                                                                                                &                                                                                                            &                                                                                  & 88\%                                                           \\ 
\cline{6-6}
                                                                                                                                                                          &                                                                                                                                                                                               &                                                                                                                &                                                                                                            &                                                                                  & \begin{tabular}[c]{@{}l@{}}premolar \\and molar\end{tabular}   \\
                                                                                                                                                                          &                                                                                                                                                                                               &                                                                                                                &                                                                                                            &                                                                                  & 82\%                                                           \\ 
\hline
\multirow{5}{*}{\begin{tabular}[c]{@{}l@{}}Automated Dental Cavity Detection\\System Using Deep \\Learning and Explainable AI\end{tabular}}                               & \multirow{5}{*}{\begin{tabular}[c]{@{}l@{}}De-identified images \\from online sources and \\consenting human participants\end{tabular}}                                                       & \multirow{5}{*}{\begin{tabular}[c]{@{}l@{}}506 \\De-identified images\end{tabular}}                            & \multirow{5}{*}{ResNet-27}                                                                                 & \multirow{5}{*}{Accuracy}                                                        & Accuracy                                                       \\
                                                                                                                                                                          &                                                                                                                                                                                               &                                                                                                                &                                                                                                            &                                                                                  & 74\%                                                           \\
                                                                                                                                                                          &                                                                                                                                                                                               &                                                                                                                &                                                                                                            &                                                                                  &                                                                \\ 
\cline{6-6}
                                                                                                                                                                          &                                                                                                                                                                                               &                                                                                                                &                                                                                                            &                                                                                  & Sensitivity                                                    \\
                                                                                                                                                                          &                                                                                                                                                                                               &                                                                                                                &                                                                                                            &                                                                                  & 87.40\%                                                        \\ 
\hline
\begin{tabular}[c]{@{}l@{}}PaXNet: Dental Caries \\Detection in Panoramic\end{tabular}                                                                                    & \begin{tabular}[c]{@{}l@{}}470 Panoramic images \\used for features extraction, \\including 240 labeled \\images for classification\end{tabular}                                              & \begin{tabular}[c]{@{}l@{}}470 \\Panoramic images\end{tabular}                                                 & PaXNet                                                                                                     & ~Accuracy                                                                        & 86.05\%                                                        \\
\hline
\end{tabular}}
\end{table}
\newpage
%\newpage\null\newpage
\chapter{DATASET}
The dataset [23] utilized in this investigation is called ODIR (Ocular Disease Intelligent Recognition). One of the most extensive public resources on Kaggle for identifying eye illnesses is this dataset. Eight classifications of ocular diseases are used to group the fundus photos in this collection. They are normal (N), myopia (M), hypertension (H), diabetes (D), cataract (C), glaucoma (G), age-related macular degeneration (A), and other abnormalities/diseases (O). The 5000 color fundus images in this dataset are split into training and testing groups. All of the photographs for this project were scaled to 224 224. The full ODIR dataset was not used for this research. Normal, cataract, glaucoma, myopia, and hypertension related images are being used. To remove imbalance, normal images are taken according to the disease’s image number to train the model. For VGG19 and Resnet50, The dataset was splitted into 90\% and 10\% ratio for training and testing. For MobilenetV2, the dataset was split into 70\%, 20\%, and 10\% ratios for training, validation, and testing. For Mobilenetv2, the accuracy was too low. That is why the dataset was augmented for better results. The number of images for each catagory is given below: 

\begin{table}[h]
\centering
\caption{Dataset Size}
\begin{tabular}{|l|l|l|} 
\hline
Class Name   & Number of Images & Augmented  \\ 
\hline
Normal       & 2100             & 2100       \\
Cataract     & 401              & 1899       \\
Glaucoma     & 396              & 1896       \\
Myopia       & 205              & 1805       \\
Hypertensive & 202              & 1702       \\
\hline
\end{tabular}
\end{table}



\newpage
%\newpage\null\newpage
\chapter{METHODOLOGY}
The broad plan and justification for your research effort are referred to as your methodology. It entails researching the theories and ideas that underpin the procedures employed in your industry in order to create a strategy that is in line with your goals.There are various types of images of ocular diseases. We chose cataracts, glaucoma, pathological myopia, and hypertensive retinopathy to work with. For each model (VGG19, Resnet50, Mobilenetv2), we tested Normal vs. Cataract, Normal vs. Glaucoma, Normal vs. Myopia, and Normal vs. Hypertension. We take 100 epoch for every model. We resize every images 224*224. We also did use some function for time reduction. EarlyStopping class stops training when a monitored metric has stopped improving. ReduceLROnPlateau class reduces learning rate when a metric has stopped improving. Here is the working steps. 
\vspace{5pt}
\begin{figure}[H]
    \centering
    \includegraphics[scale=1]{3.png}
    \caption{Working steps}
    \label{Working steps}
\end{figure}
\section{Classification Using Resnet50}
ResNet50 is a variant of the Resnet model,, which has 48 convolution layers along with 1 MaxPool and 1 Average Pool layer. It has 3.8 x 109 floating point operations. It is a widely used ResNet model, and we have explored the ResNet50 architecture in depth. So as we can see in the resnet 50 architecture contains the following element:
\begin{enumerate}
    \item A convolution with a kernel size of 7 * 7 and 64 different kernels all with a stride of size 2 giving us 1 layer.
    \item Next we see max pooling with also a stride size of 2.
    \item In the next convolution there is a 1 * 1,64 kernel following this a 3 * 3,64 kernel and at last a 1 * 1,256 kernel, These three layers are repeated in total 3 time so giving us 9 layers in this step.
    \item Next we see kernel of 1 * 1,128 after that a kernel of 3 * 3,128 and at last a kernel of 1 * 1,512 this step was repeated 4 time so giving us 12 layers in this step.
    After that there is a kernal of 1 * 1,256 and two more kernels with 3 * 3,256 and 1 * 1,1024 and this is repeated 6 time giving us a total of 18 layers.
    \item And then again a 1 * 1,512 kernel with two more of 3 * 3,512 and 1 * 1,2048 and this was repeated 3 times giving us a total of 9 layers.
    \item We do a average pool and end it with a fully connected layer containing 1000 nodes and at the end a softmax function so this gives us 1 layer.
\end{enumerate}
\begin{figure}[H]
    \centering
    \includegraphics[scale=0.5]{40_Chapter_4/resnet50.png}
    \caption{Resnet50 Architecture}
    \label{Resnet50 Architecture}
\end{figure}
\section{Classification Using VGG19}
A variation of the VGG model called VGG19 has 19 layers in total (16 convolution layers, 3 Fully connected layer, 5 MaxPool layers and 1 SoftMax layer). There are further VGG variations, including VGG11, VGG16, and others. 19.6 billion FLOPs make up VGG19. So as we can see in the VGG19architecture contains the following element:
\begin{enumerate}
    \item A fixed size of (224 * 224) RGB image was given as input to this network which means that the matrix was of shape (224,224,3).
    \item The only preprocessing that was done is that they subtracted the mean RGB value from each pixel, computed over the whole training set.
    \item Used kernels of (3 * 3) size with a stride size of 1 pixel, this enabled them to cover the whole notion of the image.
    \item spatial padding was used to preserve the spatial resolution of the image.
    \item max pooling was performed over a 2 * 2 pixel windows with sride 2.
    \item this was followed by Rectified linear unit(ReLu) to introduce non-linearity to make the model classify better and to improve computational time as the previous models used tanh or sigmoid functions this proved much better than those.
    \item implemented three fully connected layers from which first two were of size 4096 and after that a layer with 1000 channels for 1000-way ILSVRC classification and the final layer is a softmax function.
\end{enumerate}
\begin{figure}[H]
    \centering
    \includegraphics[scale=0.7]{40_Chapter_4/VGG19.jpg}
    \caption{VGG19 Architecture}
    \label{VGG19 Architecture}
\end{figure}
\vspace{5pt}
\section{Classification Using MobilenetV2}
We have explored MobileNet V2 architecture in depth. MobileNet V2 model has 53 convolution layers and 1 AvgPool with nearly 350 GFLOP. It has two main components:
\begin{enumerate}
    \item Bottleneck Residual Block
    \item Inverted Residual Block
\end{enumerate}
There are two types of Convolution layers in MobileNet V2 architecture:
\begin{enumerate}
    \item 1x1 Convolution
    \item 3x3 Depth wise Convolution 
\end{enumerate}
There are three streams and the input shape is 224x224. Our design involves a filter size of 32 for padding, a kernel size of 3, and activation function based on ReLU for the two first layers. The first max pooling layer has a pool size of 2 and strides of 2. The further plain layer combines all of the pooled characteristics into a separate cell. In the end, two thick layers were produced. The activation function for the first layer is ReLU, while the activation function for the least thick layer is softmax. The features are added to the network once they have been pre-processed. A bird's-eye perspective of the structure is shown below.
\begin{figure}[H]
    \centering
    \includegraphics[scale=0.5]{40_Chapter_4/Mb2.png}
    \caption{MobileNetV2 Architecture}
    \label{MobileNetV2 Architecture}
\end{figure}
\vspace{5pt}
\section{Evaluation}
We should be prepared with several assessment metrics to examine the classification algorithm in the event of a classification problem. As follows:
\begin{enumerate}
    \item \textbf{Confusion Matrix:} The classification model's accuracy in classifying instances into distinct groups is summarized in a table called the confusion matrix. The model's anticipated label is on one axis of the confusion matrix, while the actual label is on the other. When comparing several models, we may use the confusion matrix to assess how well each one predicted true positives (TP) and true negatives (TN). We chose a model as our basic model if it accurately predicted TP and TN compared to other models.
\begin{figure}[H]
    \centering
    \includegraphics[scale=0.7]{40_Chapter_4/4.png}
    \caption{Confusion Matrix}
    \label{Confusion Matrix}
\end{figure}
TP = True Positive (The total number of images that are
correctly detected to be positive)\\
FP = False Positive (The total number of images that are
predicted to be positive but actually are negative)\\
TN = True Negative (The number of images that are
accurately predicted to be negative)\\
FN = False Negative (The number of images that are
incorrectly predicted to be negative)\\
    \item \textbf{Precision and Recall:} Precision and recall are two metrics used to evaluate classification and retrieval systems' performance. Precision is the percentage of relevant occurrences among all retrieved examples. Recall, also known as sensitivity, is the percentage of recovered instances among all appropriate models. In a perfect classifier, precision and recall are both one. 
\newline
\begin{align*}
Recall = \frac{TP}{TP+FP} 
\\
Precision = \frac{TP}{TP+FP}
\end{align*}
\newline
    \item \textbf{Accuracy:} It is calculated by dividing the total number of correctly categorized instances by the overall number of classified examples. When the importance of each class's prediction error is equal, this measure is crucial. Here, false positives should be addressed more than false negatives.
\newline
\begin{align*}
Accuracy = \frac{TP+tn}{TP+TN+FP+FN} 
\end{align*}
\newline
    \item \textbf{F1 score:} A weighted average of recall and accuracy is the F1 score. False positive and false negative results can occur in accuracy and recall, as is well known, thus both are taken into account. In most cases, the F1 score is more helpful than accuracy, particularly if your class is distributed unevenly. When false positives and false negatives cost about the same, accuracy performs best. It is preferable to include both Precision and Recall if the costs of false positives and false negatives are significantly different.
\newline
\begin{align*}
F1 Score = \frac{2*Recall*Precision}{Recall+Precision} 
\end{align*}
\newline
    \item \textbf{Learning Curve:} For algorithms that learn (optimize their internal parameters) gradually over time, like deep learning neural networks, learning curves are frequently employed in machine learning. If maximization is the metric used to measure learning, then higher scores (bigger numbers) signify more learning. Accuracy in categorisation might serve as an example. It is more typical to employ a score that minimizes, like loss or error, where better scores (lower numbers) imply greater learning and a value of 0.0 indicates that the training dataset was learnt properly with no errors. Additionally, a hold-out validation dataset that is separate from the training dataset can be used to test it. An assessment of the validation dataset provides insight into the model's "generalizability."
\end{enumerate}
\newpage
%\newpage\null\newpage
\chapter{EXPERIMENTAL SETUPS}
After classifying the dataset into three classes: 0 (visible change without cavitation), 1(visible change with microcavitation), and 2(visible change with cavitation), the dataset was fed into the model as shown in Fig. 2. We employed three YOLO object detection models. The experimental setup for YOLOv4, YOLOv5, and YOLOv6 is described below-\\
\section{YOLOv4 Setup} Images are fed via convolutional down sampling, then supplied through a succession of layers of dense connection blocks that execute various operations and calculations.  The outputs of these blocks were then routed via a spatial pyramid pooling layer to widen receptive fields, and then through an object identification layer to identify the various classes in a picture.
\begin{figure}[H]
    \centering
    \includegraphics[scale=0.7]{50_Chapter_5/3.png}
    \caption{Working Flow}
    \label{Working Flow}
\end{figure}
Above figure shows the detailed flowchart of design and implementation from splitting dataset to evaluating result. All the necessary dependencies such as YOLOv4, CUDA, NumPy, and Python were installed from respective repositories. DarkNet [12] is a CUDA-based open-source neural network framework designed to support graphics processing units (GPU). A custom configuration file was created from the cloned YOLOv4 repository to build a custom object detector for tooth decay detection. All hyper parameters designed in the development of the custom object detector were detailed in the custom configuration file. The training module was then incorporated with a specific configuration file. Then, the model became ready to be trained with a custom dataset.\\
\textbf{Design Constraints and Parameters-Image Dataset}
\begin{enumerate}
    \item Number of Classes: 3
    \item Class name: Visible change without cavitation (0), Visible change with micro cavitation (1), Visible change with cavitation (2).
    \item Filter Size: 416*416
    \item Batch Size: 64
    \item Subdivision: 32
    \item Number of filters: (3+5)*3 = 24
\end{enumerate}
\textbf{Hyper parameters design}
\begin{enumerate}
    \item Image Size: 416*416
    \item Image channels: 3
    \item Kernel Size: 3*3
    \item Activation Function: Mish
    \item Batch Size: 64
    \item Max batches: 3*2000 = 6000
    \item Learning rate: 0.001
\end{enumerate}
The trainer's supervision was necessary for each epoch's parameter values, such as mAP, Accuracy, and Precision. To avoid model overfitting, training should be stopped after the given parameter values become constant or have very few changes. After the model was trained, best YOLOv4 weights are extracted which acts as a reference while testing the model on custom testing dataset. 
\section{YOLOv5 and Yolov6 Setup} A similar working flow as seen in fig 4.1 was used to train the dataset in YOLOv5 and YOLOv6 but without darknet. A dataset of 2618 photos was used to train YOLOv5 and Yolov6, with 86\% used for training, 9\% for validation, and 5\% for testing.
Both the model development started by installing necessary dependencies such as Python, Pytroch, YOLOv5, CUDA and roboflow from respective repositories. Datasets were uploaded and then exported into the YOLOv5 PyTorch format, which generated api keys for each dataset. It's worth mentioning that the Ultralytics solution requires a YAML file that defines where your training and test data should be stored. The Roboflow export also creates this format for us. 0 indicates visible changes without cavitation, 1 indicates visible changes with micro-cavitation, and 2 indicates visible changes with cavitation in the data.yaml file. Training configuration for YOLOv5 and YOLOv6:
\begin{table}[H]
\caption{YOLOv5 and YOLOv6 MODEL CONFIGURATIONS}
\centering
\begin{tabular}{|l|l|l|} 
\hline
\textcolor[rgb]{0.141,0.125,0.129}{Required Argument} & \textcolor[rgb]{0.141,0.125,0.129}{YOLOv5s} & \textcolor[rgb]{0.141,0.125,0.129}{YOLOv5m}  \\ 
\hline
\textcolor[rgb]{0.141,0.125,0.129}{Image Size}        & \textcolor[rgb]{0.141,0.125,0.129}{416x416} & \textcolor[rgb]{0.141,0.125,0.129}{416x416}  \\ 
\hline
\textcolor[rgb]{0.141,0.125,0.129}{Batch Size}        & \textcolor[rgb]{0.141,0.125,0.129}{16}      & \textcolor[rgb]{0.141,0.125,0.129}{32}       \\ 
\hline
\textcolor[rgb]{0.141,0.125,0.129}{Epoch}             & \textcolor[rgb]{0.141,0.125,0.129}{100}     & \textcolor[rgb]{0.141,0.125,0.129}{100}      \\
\hline
\end{tabular}
\end{table}
Training losses and performance data was saved to a log file created before with the —name flag when the model was trained. The weight values were saved in .pt files. On test photos, the best weights was applied and got somewhat improved accuracy in YOLOv5 and YOLOv6.

\newpage
\chapter{DEPLOYMENT}
The process of integrating a machine learning model into an already-existing production environment is known as deployment, and it allows you to use data to make useful business choices. It can be one of the most challenging stages of the machine learning life cycle and is one of the final ones. Frequently, traditional model-building languages are incompatible with an organization's IT systems, requiring data scientists and programmers to spend considerable time and brainpower rebuilding them. A model must be successfully put into production before it can be used for making useful decisions. The effect of your model will be much diminished if you are unable to consistently derive useful insights from it.Machine learning models must be smoothly deployed into production in order for businesses to use them to start making useful judgments. This will maximize their worth.\\ 

\begin{figure}[H]
    \centering
    \includegraphics[scale=0.4]{70_Deployment/Deploy.png}
    \caption{Deployment}
    \label{deploy}
\end{figure}

A Python-based open source app framework is called Streamlit. It enables us to quickly develop web applications for data science and machine learning. Major Python libraries like scikit-learn, Keras, PyTorch, SymPy (latex), NumPy, pandas, and Matplotlib are all compatible with it. We used Streamlit to deploy our project.
We use Sqlite3 for temporary Data Management. Python SQLite3 module is used to integrate the SQLite database with Python. It is a standardized Python DBI API 2.0 and provides a straightforward and simple-to-use interface for interacting with SQLite databases. There is no need to install this module separately as it comes along with Python after the 2.5x version.

\newpage
%\newpage\null\newpage
\chapter{CONCLUSION}

Early identification of tooth decay can save costs on dental treatments and even reverse the decay process. A deep learning strategy to detect tooth decay is presented in this paper. The dataset was trained in two different object detection algorithms to see which one performed better. YOLOv6 was more accurate but YOLOv5 was faster in this condition. YOLOv4 was good but YOLOv5 and YOLOv6 was better in every angle. The YOLOv6 model had the highest accuracy of 99.2\%, while the YOLOv4 model had the lowest accuracy of 98.6\%.\\ 
In comparison to the current, CNN-based dental caries classification models, This suggested technique can produce results that are more spectacular while using less computing power. The nicest thing about our suggested approach is how easy it may be applied to various kinds of illness categorization based on medical images. Such a method will change the area of visual illness diagnostics and be of enormous use to medical Experts. Additionally, this work could benefit from the use of ocular image segmentation.  Additionally, a system like this would revolutionize the field of diagnosing dental caries and be very helpful to medical professionals. my opinion is that it can still be a valuable model, and there will likely be possibilities to improve it with further research and study in the near future.


\newpage

%%%%%% Appendices %%%%%%
\appendix
\renewcommand{\chaptername}{Appendix}
\addtocontents{toc}{\protect\renewcommand{\protect\cftchappresnum}{\chaptername~}}
%%%%% Creating Appendix A
\chapter{CODE}
\label{appendix_0}
\graphicspath{{100_Appendices/}}

\section{Deployment Code}
\begin{lstlisting} [language=Python, caption=app.py]
import streamlit as st
import pandas as pd

# Security
#passlib,hashlib,bcrypt,scrypt
import hashlib
def make_hashes(password):
    return hashlib.sha256(str.encode(password)).hexdigest()

def check_hashes(password,hashed_text):
    if make_hashes(password) == hashed_text:
        return hashed_text
    return False
# DB Management
import sqlite3 
conn = sqlite3.connect('data.db')
c = conn.cursor()
# DB  Functions
def create_usertable():
    c.execute('CREATE TABLE IF NOT EXISTS 
    userstable(username TEXT,password TEXT)')

def add_userdata(username,password):
    c.execute('INSERT INTO userstable(username,password) 
    VALUES (?,?)',(username,password))
    conn.commit()

def login_user(username,password):
    c.execute('SELECT * FROM userstable 
    WHERE username =? AND password = ?',(username,password))
    data = c.fetchall()
    return data

def view_all_users():
    c.execute('SELECT * FROM userstable')
    data = c.fetchall()
    return data

def main():
    """OCULAR DISEASES RECONGNITION SYSTEM"""
    st.title("OCULAR DISEASES RECONGNITION SYSTEM")
    menu = ["Login","SignUp"]
    choice = st.sidebar.selectbox("Menu",menu)
    if choice == "Login":
        username = st.sidebar.text_input("User Name")
        password = st.sidebar.text_input("Password",type='password')
        if st.sidebar.checkbox("Login"):
            # if password == '12345':
            create_usertable()
            hashed_pswd = make_hashes(password)
            result= 
            login_user(username,check_hashes(password,hashed_pswd))
            if result:
                st.success("Logged In as {}".format(username))
                # Custom imports 
                import tensorflow as tf
                import numpy as np
                from PIL import Image, ImageOps
                st.title("Image Classification")
                upload_file = st.sidebar.file_uploader(
                "Upload images", type = 'jpg')
                generate_pred = st.sidebar.button("predict")
                model = tf.keras.models.load_model('CVN.h5')
                def import_n_pred(image_data, model):
                    size = (224,224)
                    image = ImageOps.fit
                    (image_data, size, Image.ANTIALIAS)
                    img = np.asarray(image)
                    reshape = img[np.newaxis,...]
                    pred = model.predict(reshape)
                    return pred
                if generate_pred:
                    image = Image.open(upload_file)
                    with st.expander('image', expanded=True):
                        st.image(image, use_column_width=True)
                    pred = import_n_pred(image, model)
                    labels = ['Cataract' , 'Normal']
                    st.title("The result in CVN model is {}"
                    .format(labels[np.argmax(pred)]))
                    if (np.argmax(pred)==1):
                        model = tf.keras.models.load_model
                        ('GVN.h5')
                        pred = import_n_pred(image, model)
                        labels = ['Glaucoma' , 'Normal']
                        st.title("The result in GVN model is {}"
                        .format(labels[np.argmax(pred)]))
                        if (np.argmax(pred)==1):
                            model = tf.keras.models.load_model
                            ('MVN.h5')
                            pred = import_n_pred(image, model)
                            labels = ['Myopia' , 'Normal']
                            st.title("The result in MVN model is {}"
                            .format(labels[np.argmax(pred)]))
                            if (np.argmax(pred)==1):
                                model = tf.keras.models.load_model
                                ('HVN.h5')
                                pred = import_n_pred(image, model)
                                labels = ['Hypertensive','Normal']
                                st.title("The result in HVN model is {}"
                                .format(labels[np.argmax(pred)]))
            else:
                st.warning("Incorrect Username/Password")

    elif choice == "SignUp":
        st.subheader("Create New Account")
        new_user = st.text_input("Username")
        new_password = st.text_input("Password",type='password')

        if st.button("Signup"):
            create_usertable()
            add_userdata(new_user,make_hashes(new_password))
            st.success("You have successfully created a valid Account")
            st.info("Go to Login Menu to login")

if __name__ == '__main__':   	
    main()
\end{lstlisting}
\newpage
% \null\newpage

\setstretch{1}

\vspace{12pt}
\begin{thebibliography}{00}
\bibitem{b1} Bourne, R. R., Stevens, G. A., White, R. A., Smith, J. L., Flaxman, S. R., Price, H., Jonas, J. B., Keeffe, J., Leasher, J., Naidoo, K., Pesudovs, K., Resnikoff, S., & Taylor, H. R. (2013). Causes of vision loss worldwide, 1990–2010: a systematic analysis. The Lancet Global Health, 1(6). https://doi.org/10.1016/s2214-109x(13)70113-x 
\bibitem{b2} Sommer, A., Tielsch, J. M., Katz, J., Quigley, H. A., Gottsch, J. D., Javitt, J. C., Martone, J. F., Royall, R. M., Witt, K. A., & Ezrine, S. (1991). Racial Differences in the Cause-Specific Prevalence of Blindness in East Baltimore. New England Journal of Medicine, 325(20), 1412–1417. https://doi.org/10.1056/nejm199111143252004 
\bibitem{b3} Congdon, N., O'Colmain, B., Klaver, C. C., Klein, R., Muñoz, B., Friedman, D. S., Kempen, J., Taylor, H. R., Mitchell, P., & Eye Diseases Prevalence Research Group (2004). Causes and prevalence of visual impairment among adults in the United States. Archives of ophthalmology (Chicago, Ill. : 1960), 122(4), 477–485. 
\bibitem{b4} Application of Ocular Fundus Photography and Angiography. (2014). Ophthalmological Imaging and Applications, 154–175. https://doi.org/10.1201/b17026-12 
\bibitem{b5} Rowe, S., MacLean, C. H., & Shekelle, P. G. (2004). Preventing Visual Loss From Chronic Eye Disease in Primary Care. JAMA, 291(12), 1487. https://doi.org/10.1001/jama.291.12.1487 
\bibitem{b6} Kessel, L., Erngaard, D., Flesner, P., Andresen, J., Tendal, B., & Hjortdal, J. (2015). Cataract surgery and age‐related macular degeneration. An evidence‐based update. Acta Ophthalmologica, 93(7), 593–600. https://doi.org/10.1111/aos.12665Li, N., Li, T., Hu, C., Wang, K.
\bibitem{b7} Li, N., Li, T., Hu, C., Wang, K., & Kang, H. (2021). A Benchmark of Ocular Disease Intelligent Recognition: One Shot for Multi-disease Detection. Benchmarking, Measuring, and Optimizing, 177–193. https://doi.org/10.1007/978-3-030-71058-3-11
\bibitem{b8}K. N. Alam, M. S. Khan, A. R. Dhruba et al., “Deep learning-based sentiment analysis of COVID-19 vaccination responses from Twitter data,” Computational and Mathematical Methods in Medicine, vol. 2021, pp. 1–15, 2021.
\bibitem{b9}J. He, C. Li, J. Ye, Y. Qiao, and L. Gu, “Self-speculation of clinical features based on knowledge distillation for accurate ocular disease classification,” Biomedical Signal Processing and Control, vol. 67, Article ID 102491, 2021.
\bibitem{b10}K. N. Alam and M. M. Khan, “CNN based COVID-19 prediction from chest X-ray images,” in Proceedings of the 2021 IEEE 12th Annual Ubiquitous Computing, Electronics & Mobile Communication Conference (UEMCON), pp. 0486–0492, IEEE, New York, USA, 1 December 2021.
\bibitem{b11}A. G. Roy, S. Conjeti, S. P. K. Karri et al., “ReLayNet: retinal layer and fluid segmentation of macular optical coherence tomography using fully convolutional networks,” Biomedical Optics Express, vol. 8, no. 8, pp. 3627–3642, 2017.
\bibitem{b12}C. S. Lee, A. J. Tyring, N. P. Deruyter, Y. Wu, A. Rokem, and A. Y. Lee, “Deep-learning based, automated segmentation of macular edema in optical coherence tomography,” Biomedical Optics Express, vol. 8, no. 7, pp. 3440–3448, 2017.
\bibitem{b13}S. P. K. Karri, D. Chakraborty, and J. Chatterjee, “Transfer learning based classification of optical coherence tomography images with diabetic macular edema and dry age-related macular degeneration,” Biomedical Optics Express, vol. 8, no. 2, pp. 579–592, 2017.
\bibitem{b14}M. Oda, T. Yamaguchi, H. Fukuoka, Y. Ueno, and K. Mori, “Automated Eye Disease Classification Method from Anterior Eye Image Using Anatomical Structure Focused Image Classification Technique,” 2020, https://arxiv.org/abs/2005.01433.
\bibitem{b15}F. Eperjesi, C. W. Fowler, and A. J. Kempster, “Luminance and chromatic contrast effects on reading and object recognition in low vision: a review of the literature,” Ophthalmic and Physiological Optics, vol. 15, no. 6, pp. 561–568, 1995.
\bibitem{b16}N. M. Dipu, S. Alam Shohan, and K. M. A. Salam, “Ocular disease detection using advanced neural network based classification algorithms,” ASIAN JOURNAL OF CONVERGENCE IN TECHNOLOGY, vol. 7, no. 2, pp. 91–99, 2021.
\bibitem{b17}M. S. Khan et al., “Deep learning for ocular disease recognition: An inner-class balance,” Comput. Intell. Neurosci., vol. 2022, p. 5007111, 2022.
\bibitem{b18} L. Jain, H. V. S. Murthy, C. Patel and D. Bansal, "Retinal Eye Disease Detection Using Deep Learning," 2018 Fourteenth International Conference on Information Processing (ICINPRO), 2018, pp. 1-6, doi: 10.1109/ICINPRO43533.2018.9096838.
\bibitem{b19} Nazir, T.; Irtaza, A.; Javed, A.; Malik, H.; Hussain, D.; Naqvi, R.A. Retinal Image Analysis for Diabetes-Based Eye Disease Detection Using Deep Learning. Appl. Sci. 2020, 10, 6185. https://doi.org/10.3390/app10186185
\bibitem{b20}Sarki, R., Ahmed, K., Wang, H. et al. Automated detection of mild and multi-class diabetic eye diseases using deep learning. Health Inf Sci Syst 8, 32 (2020). https://doi.org/10.1007/s13755-020-00125-5
\bibitem{b21} J. Shan and L. Li, "A Deep Learning Method for Microaneurysm Detection in Fundus Images," 2016 IEEE First International Conference on Connected Health: Applications, Systems and Engineering Technologies (CHASE), 2016, pp. 357-358, doi: 10.1109/CHASE.2016.12.
\bibitem{b22} Malik, S.; Kanwal, N.; Asghar, M.N.; Sadiq, M.A.A.; Karamat, I.; Fleury, M. Data Driven Approach for Eye Disease Classification with Machine Learning. Appl. Sci. 2019, 9, 2789. https://doi.org/10.3390/app9142789
\bibitem{b23} “Ocular disease recognition,” https://www.kaggle.com/andrewmvd/ocular-disease-recognition-odir5k.
\end{thebibliography}
\vspace{12pt}
\end{document}
