\chapter*{\centering Abstract}

\pagestyle{fancy}
\fancyhf{}
\fancyhead[LO,RE]{Abstract}
\fancyhead[LE,RO]{\thepage}

\addcontentsline{toc}{chapter}{Abstract}

Cavities, commonly known as tooth decay, are portions of the hard surface of teeth that have been irreversibly damaged. It eventually turns into tiny gaps or holes. It can cause infection, pain, and tooth loss if not treated appropriately. For developing and underdeveloped countries, regular dental appointments can be costly. This study aimed to develop and evaluate the performance of a tooth decay detection system that used a deep learning technique based on a convolutional neural network (CNN) to detect tooth decay from oral photographs. We used a collection of 233 pictures of teeth with cavitation. Then, we augmented the dataset and used Roboflow to manage our dataset. The input was clear photos of affected teeth on a white background. After some pre-processing, the dataset was trained on three separate object detection models – YOLOv4tiny, YOLOv5s, and YOLOv6. All of them were evaluated by mean average precision. The mAP@.5 for   YOLOv4tiny, YOLOv5s, and YOLOv6 are respectively 98.68\%, 98.9\%, and 99.25\%.\\
\textbf{Index Terms: }Dataset, tooth decay detection, deep learning, transfer learning.

