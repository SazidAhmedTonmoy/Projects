
\chapter{LITERATURE REVIEWS}
% \graphicspath{{Chapter_2/Vector/}{Chapter_2/}}

An overview, a summary, and an assessment of the state of knowledge in a particular field of study make up a literature review. It could also highlight methodological concerns and make recommendations for further study.In this chapter, we have reviewed some papers related to our work.


\section{Deep Learning-Based Dental Plaque Detection on Primary Teeth: A Comparison With Clinical Assessments} Wenzhe You and his team aimed to train a deep convolutional neural network (CNN) to detect caries lesions using DeepLabV3+.  886 intraoral images of primary teeth were utilized to train a conventional neural network (CNN) architecture. 98 intraoral pictures of primary teeth were evaluated by the AI model to verify clinical viability. Additionally, digital camera images of teeth were taken. A skilled pediatric dentist evaluated the images and noted the plaque-containing areas. The plaque-containing sites were then detected after the application of a plaque-disclosing agent. To assess the consistency of the manual diagnosis, the dentist drew the plaque area on the 98 digital camera photographs once again after one week. To assess the diagnostic effectiveness of each method based on lower-resolution pictures, 102 intraoral photographs of primary teeth were annotated to signify the plaque regions acquired by the AI model and the dentist. The degree of detection accuracy was measured using the mean intersection-over-union (MIoU) metric.\\
To start, they used the visual object classes dataset to pretrain the fundamental DeepLab network and derive the initial weights using transfer learning methods. Second, they used their picture dataset of primary teeth, which includes images of 886 primary teeth before and after employing a dental plaque-disclosing agent, to train a DeepLabV3+ model. To enable the AI model to compare the findings and learn from its errors, the dental plaque detected by the AI model was compared with the actual dental plaque regions.\\
The MIoU for recognizing plaque on the examined dental pictures was 0.726 ± 0.165. The dentist's MIoU when first diagnosing the 98 photographs obtained by the digital camera was 0.695 ± 0.269 and 0.689 ± 0.253 after one week. The AI model had a higher MIoU (0.736 ± 0.174) than the dentist, and the outcomes were unchanged after one week. The MIoU was 0.652 ± 0.195 for the dentist and 0.724 ± 0.159 for the AI model when they evaluated the 102 intraoral pictures. A paired t-test revealed no statistically significant difference between the human expert and AI model in the ability to identify dental plaque on primary teeth (P >.05). [5]\\

\section{Deep Learning Application in Dental Caries Detection Using Intraoral Photos Taken by Smartphones} Mai Thi Giang Thanh and his team worked with Intraoral Photos Taken by Smartphones to train several deep convolutional neural network (CNN) to detect dental caries. The objective of this work was to use a deep learning system to diagnose the phases of smooth surface caries using photos from a smartphone. Materials and procedures A training dataset of 1902 images of teeth's smooth surface acquired using an iPhone 7 by 695 individuals was used. To identify early caries lesions and cavities, four deep learning models—You Only Look Once version 3 (YOLOv3), RetinaNet, Faster Region-Based Convolutional Neural Networks (Faster R-CNNs), and Single-Shot Multi-Box Detector (SSD)—were examined. The International Caries Categorization and Management System (ICCMS) classification of a dentist's diagnosis based on an image inspection served as the reference standard.\\
The two evaluated models with the highest sensitivity for cavitated caries were YOLOv3 and Faster R-CNN, with 87.4\% and 71.44\%, respectively. For visibly non-cavitated samples, these two models' sensitivity levels were only 36.9\% and 26\%, respectively (VNC). For cavitated caries and over 71\% for VNC, the specificity of the four models was above 86\%. For the clinical diagnosis of dental caries using smartphone photos, the YOLOv3 and Faster R-CNN models showed promise. The new work offers a rudimentary understanding of how AI may be applied in clinical settings once it has been developed in the lab. [6]\\

\section{Detection and Diagnosis of Dental Caries Using a Deep Learning-Based Convolutional Neural Network Algorithm} Jae-Hong Lee and his team worked with dental images to evaluate the efficacy of deep CNN algorithms for detection and diagnosis of dental caries on periapical radiographs. A training and validation dataset (n = 2400 [80\%]) and a test dataset (n = 600 [20\%]) were created from a total of 3000 periapical radiography images. For preprocessing and transfer learning, a GoogLeNet Inception v3 CNN network that has already been trained was employed. For detection and diagnostic performance of the deep CNN algorithm, the diagnostic accuracy, sensitivity, specificity, positive predictive value, negative predictive value, receiver operating characteristic (ROC) curve, and area under the curve (AUC) were calculated. \\
Premolar, molar, and premolar and molar models all had diagnostic accuracies of 89.0 percent (80.4-93.3), 88.0 percent (79.2-93.1), and 82.0 percent (75.5-87.1), respectively. The AUC on premolar, molar, and combined premolar and molar models for the deep CNN method was 0.917 (95 percent CI 0.860-0.975), 0.890 (95 percent CI 0.819-0.961), and 0.845 (95 percent CI 0.790-0.901). The best AUC was produced by the premolar model, which was far better than that for the other models (P 0.001). This study showed how deep CNN architecture might be useful for identifying and diagnosing dental cavities. In periapical radiographs, a deep CNN algorithm significantly improved performance in identifying dental caries.[7]\\
\section{Automated Dental Cavity Detection System Using Deep Learning and Explainable AI}
An artificial intelligence system that identifies the existence of dental cavities on photos and visually explains each diagnostic was created to manage tooth cavities. The technique used in this work identifies cavities on pictures of several teeth and four tooth surfaces, unlike earlier systems that could only detect cavities on one removed tooth with one exposed tooth surface. 506 de-identified photos from web sources and willing human subjects were gathered for training. A ResNet-27 design was found to be the most effective using curriculum learning, reaching 82.8\% accuracy and 1.0 in sensitivity. The system's diagnosis might also be visually explained using Local Interpretable Model Agnostic Explanation. This technology can clearly explain its diagnosis to users, which is an essential ability used by dentists.[8]
\section{PaXNet: Dental Caries Detection in Panoramic X-ray using Ensemble Transfer Learning and Capsule Classifier}
The proposed model benefits from various pretrained deep learning models through transfer learning to extract relevant features from x-rays and uses a capsule network to draw prediction results. On a dataset of 470 Panoramic images used for features extraction, including 240 labeled images for classification, their model achieved an accuracy score of 86.05\% on the test set. As long as the difficulties of employing Panoramic x-rays of actual patients are taken into consideration, the resultant score reveals satisfactory detection performance and an increase in caries detection time. Their model achieved recall scores of 69.44\% and 90.52\% for moderate and severe caries lesions, respectively, in the test set of photos containing caries lesions, demonstrating that it is easier to detect severe caries spots and that effective mild caries identification requires a more robust and bigger dataset. This work is a step toward creating a completely automated effective decision support system to help domain specialists, especially in light of the originality of the current research study's use of panoramic photographs.[9]\\
\begin{table}[H}
\caption{Literature Review}
\centering
\resizebox{\textwidth}{!}{\begin{tabular}{|l|l|l|l|l|l|} 
\hline
Work                                                                                                                                                                      & Dataset Features                                                                                                                                                                              & Dataset Size                                                                                                   & \begin{tabular}[c]{@{}l@{}}Algorithm \\Selection\end{tabular}                                              & Evaluation                                                                       & Results                                                        \\ 
\hline
\multirow{3}{*}{\begin{tabular}[c]{@{}l@{}}Deep learning Based Dental Plaque\\Detection on Primary Teeth: A Comparison\\with Clinical Assessments\end{tabular}}           & Digital camera images of teeth,                                                                                                                                                               & \multirow{3}{*}{\begin{tabular}[c]{@{}l@{}}886 training data , \\102 testing data\end{tabular}}                & \multirow{3}{*}{DeepLabV3+}                                                                                & \multirow{3}{*}{\begin{tabular}[c]{@{}l@{}}\\(MIoU) metric.\end{tabular}}        & \multirow{3}{*}{0.724 ±~0.159}                                 \\
                                                                                                                                                                          & \begin{tabular}[c]{@{}l@{}}Annotated plaque area, \\Dataset evaluated both \\human expert and AI\end{tabular}                                                                                 &                                                                                                                &                                                                                                            &                                                                                  &                                                                \\
                                                                                                                                                                          &                                                                                                                                                                                               &                                                                                                                &                                                                                                            &                                                                                  &                                                                \\ 
\hline
\multirow{4}{*}{\begin{tabular}[c]{@{}l@{}}Deep Learning Application in Dental Caries\\Detection Using Intraoral Photos \\Taken by Smartphones\end{tabular}}              & \begin{tabular}[c]{@{}l@{}}Smartphone photos \\taken with Iphone7,\end{tabular}                                                                                                               & \multirow{4}{*}{1902 training dataset}                                                                         & \multirow{4}{*}{\begin{tabular}[c]{@{}l@{}}YOLOv3, \\RetinaNet, \\Faster \\R-CNNs, \\and SSD\end{tabular}} & \multirow{4}{*}{\begin{tabular}[c]{@{}l@{}}Accuracy, \\Sensitivity\end{tabular}} & \begin{tabular}[c]{@{}l@{}}Highest \\Accuracy\end{tabular}     \\
                                                                                                                                                                          & \begin{tabular}[c]{@{}l@{}}Images of teeth’s \\smooth surface, \\Annotated area\end{tabular}                                                                                                  &                                                                                                                &                                                                                                            &                                                                                  & \begin{tabular}[c]{@{}l@{}}YOLOv3 \\74\%\end{tabular}          \\ 
\cline{6-6}
                                                                                                                                                                          &                                                                                                                                                                                               &                                                                                                                &                                                                                                            &                                                                                  & \begin{tabular}[c]{@{}l@{}}Highest \\Sensitivity\end{tabular}  \\
                                                                                                                                                                          &                                                                                                                                                                                               &                                                                                                                &                                                                                                            &                                                                                  & \begin{tabular}[c]{@{}l@{}}R-CNN \\87.4\%\end{tabular}         \\ 
\hline
\multirow{6}{*}{\begin{tabular}[c]{@{}l@{}}Detection and Diagnosis of Dental\\Caries Using a Deep Learning \\Based Convolutional Neural Network \\Algorithm\end{tabular}} & \multirow{6}{*}{\begin{tabular}[c]{@{}l@{}}Images of dental caries \\on periapical radiographs, \\Annotated area, \\Premolar, molar, and \\premolar and \\molar are the classes\end{tabular}} & \multirow{6}{*}{\begin{tabular}[c]{@{}l@{}}3000 images, \\2400 training data, \\600 testing data\end{tabular}} & \multirow{6}{*}{\begin{tabular}[c]{@{}l@{}}GoogLeNet \\Inception v3\end{tabular}}                          & \multirow{6}{*}{Accuracy}                                                        & Premolar                                                       \\
                                                                                                                                                                          &                                                                                                                                                                                               &                                                                                                                &                                                                                                            &                                                                                  & 89\%                                                           \\ 
\cline{6-6}
                                                                                                                                                                          &                                                                                                                                                                                               &                                                                                                                &                                                                                                            &                                                                                  & Molar                                                          \\
                                                                                                                                                                          &                                                                                                                                                                                               &                                                                                                                &                                                                                                            &                                                                                  & 88\%                                                           \\ 
\cline{6-6}
                                                                                                                                                                          &                                                                                                                                                                                               &                                                                                                                &                                                                                                            &                                                                                  & \begin{tabular}[c]{@{}l@{}}premolar \\and molar\end{tabular}   \\
                                                                                                                                                                          &                                                                                                                                                                                               &                                                                                                                &                                                                                                            &                                                                                  & 82\%                                                           \\ 
\hline
\multirow{5}{*}{\begin{tabular}[c]{@{}l@{}}Automated Dental Cavity Detection\\System Using Deep \\Learning and Explainable AI\end{tabular}}                               & \multirow{5}{*}{\begin{tabular}[c]{@{}l@{}}De-identified images \\from online sources and \\consenting human participants\end{tabular}}                                                       & \multirow{5}{*}{\begin{tabular}[c]{@{}l@{}}506 \\De-identified images\end{tabular}}                            & \multirow{5}{*}{ResNet-27}                                                                                 & \multirow{5}{*}{Accuracy}                                                        & Accuracy                                                       \\
                                                                                                                                                                          &                                                                                                                                                                                               &                                                                                                                &                                                                                                            &                                                                                  & 74\%                                                           \\
                                                                                                                                                                          &                                                                                                                                                                                               &                                                                                                                &                                                                                                            &                                                                                  &                                                                \\ 
\cline{6-6}
                                                                                                                                                                          &                                                                                                                                                                                               &                                                                                                                &                                                                                                            &                                                                                  & Sensitivity                                                    \\
                                                                                                                                                                          &                                                                                                                                                                                               &                                                                                                                &                                                                                                            &                                                                                  & 87.40\%                                                        \\ 
\hline
\begin{tabular}[c]{@{}l@{}}PaXNet: Dental Caries \\Detection in Panoramic\end{tabular}                                                                                    & \begin{tabular}[c]{@{}l@{}}470 Panoramic images \\used for features extraction, \\including 240 labeled \\images for classification\end{tabular}                                              & \begin{tabular}[c]{@{}l@{}}470 \\Panoramic images\end{tabular}                                                 & PaXNet                                                                                                     & ~Accuracy                                                                        & 86.05\%                                                        \\
\hline
\end{tabular}}
\end{table}